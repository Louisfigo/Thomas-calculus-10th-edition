\documentclass[a4paper, 12pt]{article} % Font size (can be 10pt, 11pt or 12pt) and paper size (remove a4paper for US letter paper)
\usepackage{amsmath,amsfonts,bm}
\usepackage{hyperref,verbatim}
\usepackage{amsthm,epigraph} 
\usepackage{amssymb}
\usepackage{framed,mdframed}
\usepackage{graphicx,color} 
\usepackage{mathrsfs,xcolor} 
\usepackage[all]{xy}
\usepackage{fancybox} 
% \usepackage{xeCJK}
\usepackage{CJKutf8}
\newtheorem*{adtheorem}{定理}
% \setCJKmainfont[BoldFont=FZYaoTi,ItalicFont=FZYaoTi]{FZYaoTi}
\definecolor{shadecolor}{rgb}{1.0,0.9,0.9} %背景色为浅红色
\newenvironment{theorem}
{\bigskip\begin{mdframed}[backgroundcolor=gray!40,rightline=false,leftline=false,topline=false,bottomline=false]\begin{adtheorem}}
    {\end{adtheorem}\end{mdframed}\bigskip}
\newtheorem*{bdtheorem}{定义}
\newenvironment{definition}
{\bigskip\begin{mdframed}[backgroundcolor=gray!40,rightline=false,leftline=false,topline=false,bottomline=false]\begin{bdtheorem}}
    {\end{bdtheorem}\end{mdframed}\bigskip}
\newtheorem*{cdtheorem}{习题}
\newenvironment{exercise}
{\bigskip\begin{mdframed}[backgroundcolor=gray!40,rightline=false,leftline=false,topline=false,bottomline=false]\begin{cdtheorem}}
    {\end{cdtheorem}\end{mdframed}\bigskip}
\newtheorem*{ddtheorem}{注}
\newenvironment{remark}
{\bigskip\begin{mdframed}[backgroundcolor=gray!40,rightline=false,leftline=false,topline=false,bottomline=false]\begin{ddtheorem}}
    {\end{ddtheorem}\end{mdframed}\bigskip}
\newtheorem*{edtheorem}{引理}
\newenvironment{lemma}
{\bigskip\begin{mdframed}[backgroundcolor=gray!40,rightline=false,leftline=false,topline=false,bottomline=false]\begin{edtheorem}}
    {\end{edtheorem}\end{mdframed}\bigskip}
\newtheorem*{pdtheorem}{例}
\newenvironment{example}
{\bigskip\begin{mdframed}[backgroundcolor=gray!40,rightline=false,leftline=false,topline=false,bottomline=false]\begin{pdtheorem}}
    {\end{pdtheorem}\end{mdframed}\bigskip}

\usepackage[protrusion=true,expansion=true]{microtype} % Better typography
\usepackage{wrapfig} % Allows in-line images
\usepackage{mathpazo} % Use the Palatino font
\usepackage[T1]{fontenc} % Required for accented characters
\linespread{1.05} % Change line spacing here, Palatino benefits from a slight increase by default

\makeatletter
\renewcommand\@biblabel[1]{\textbf{#1.}} % Change the square brackets for each bibliography item from '[1]' to '1.'
\renewcommand{\@listI}{\itemsep=0pt} % Reduce the space between items in the itemize and enumerate environments and the bibliography

\renewcommand{\maketitle}{ % Customize the title - do not edit title
  % and author name here, see the TITLE block
  % below
  \renewcommand\refname{参考文献}
  \newcommand{\D}{\displaystyle}\newcommand{\ri}{\Rightarrow}
  \newcommand{\ds}{\displaystyle} \renewcommand{\ni}{\noindent}
  \newcommand{\pa}{\partial} \newcommand{\Om}{\Omega}
  \newcommand{\om}{\omega} \newcommand{\sik}{\sum_{i=1}^k}
  \newcommand{\vov}{\Vert\omega\Vert} \newcommand{\Umy}{U_{\mu_i,y^i}}
  \newcommand{\lamns}{\lambda_n^{^{\scriptstyle\sigma}}}
  \newcommand{\chiomn}{\chi_{_{\Omega_n}}}
  \newcommand{\ullim}{\underline{\lim}} \newcommand{\bsy}{\boldsymbol}
  \newcommand{\mvb}{\mathversion{bold}} \newcommand{\la}{\lambda}
  \newcommand{\La}{\Lambda} \newcommand{\va}{\varepsilon}
  \newcommand{\be}{\beta} \newcommand{\al}{\alpha}
  \newcommand{\dis}{\displaystyle} \newcommand{\R}{{\mathbb R}}
  \newcommand{\N}{{\mathbb N}} \newcommand{\cF}{{\mathcal F}}
  \newcommand{\gB}{{\mathfrak B}} \newcommand{\eps}{\epsilon}
  \begin{flushright} % Right align
    {\LARGE\@title} % Increase the font size of the title
    
    \vspace{50pt} % Some vertical space between the title and author name
    
    {\large\@author} % Author name
    \\\@date % Date
    
    \vspace{40pt} % Some vertical space between the author block and abstract
  \end{flushright}
}

% ----------------------------------------------------------------------------------------
%	TITLE
% ----------------------------------------------------------------------------------------
\begin{document}
\begin{CJK}{UTF8}{gkai}
  \title{\textbf{四维单位球的体积}}
  % \setlength\epigraphwidth{0.7\linewidth}
  \author{\small{叶卢庆}\\{\small{杭州师范大学理学院,学
        号:1002011005}}\\{\small{Email:h5411167@gmail.com}}} % Institution
  \renewcommand{\today}{\number\year. \number\month. \number\day}
  \date{\today} % Date
  
  % ----------------------------------------------------------------------------------------
  
  
  \maketitle % Print the title section
  
  % ----------------------------------------------------------------------------------------
  %	ABSTRACT AND KEYWORDS
  % ----------------------------------------------------------------------------------------
  
  % \renewcommand{\abstractname}{摘要} % Uncomment to change the name of the abstract to something else
  
  % \begin{abstract}
  
  % \end{abstract}
  
  % \hspace*{3,6mm}\textit{关键词:} % Keywords
  
  % \vspace{30pt} % Some vertical space between the abstract and first section
  
  % ----------------------------------------------------------------------------------------
  %	ESSAY BODY
  % ----------------------------------------------------------------------------------------
  \begin{exercise}[12.28(超体积)]
    求四维单位球 $x^2+y^2+z^2+w^2\leq 1$ 的体积.
  \end{exercise}
  \begin{proof}[解]
    易得积分为
    \begin{align*}
      \int_{-1}^1\int_{-\sqrt{1-w^2}}^{\sqrt{1-w^2}}\int_{-\sqrt{1-w^2-x^2}}^{\sqrt{1-w^2-x^2}}\int_{-\sqrt{1-w^2-x^2-y^2}}^{\sqrt{1-w^2-x^2-y^2}}dzdydxdw.
    \end{align*}
    我们进行球坐标变量替换.令
$$
w=\rho\cos\xi,z=\rho\sin\xi\cos\phi,y=\rho\sin\xi\sin\phi\sin\theta,x=\rho\sin\xi\sin\phi\cos\theta.
$$
其中 $\xi\in [0,\pi],\phi\in [0,\pi],\theta\in [0,2\pi)$.则我们来看Jacobi 行列式
\begin{align*}
  \begin{vmatrix}
    \frac{\pa x}{\pa \rho}&\frac{\pa x}{\pa \xi}&\frac{\pa x}{\pa
      \phi}&\frac{\pa x}{\pa\theta}\\
    \frac{\pa y}{\pa \rho}&\frac{\pa y}{\pa \xi}&\frac{\pa y}{\pa
      \phi}&\frac{\pa y}{\pa\theta}\\
    \frac{\pa z}{\pa \rho}&\frac{\pa z}{\pa \xi}&\frac{\pa z}{\pa
      \phi}&\frac{\pa z}{\pa\theta}\\
    \frac{\pa w}{\pa \rho}&\frac{\pa w}{\pa \xi}&\frac{\pa w}{\pa
      \phi}&\frac{\pa w}{\pa\theta}\\
  \end{vmatrix}&=\begin{vmatrix}
    \sin\xi\sin\phi\cos\theta&\rho\cos\xi\sin\phi\cos\theta&\rho\sin\xi\cos\phi\cos\theta&-\rho\sin\xi\sin\phi\sin\theta\\
    \sin\xi\sin\phi\sin\theta&\rho\cos\xi\sin\phi\sin\theta&\rho\sin\xi\cos\phi\sin\theta&\rho\sin\xi\sin\phi\cos\theta\\
    \sin\xi\cos\phi&\rho\cos\xi\cos\phi&-\rho\sin\xi\sin\phi&0\\
    \cos\xi&-\rho\sin\xi&0&0\\
  \end{vmatrix}
\end{align*}
下面我们来算.先把一些该提取的给提取掉,得到
\begin{align*}
  \rho^3\sin^2\xi\sin\phi
  \begin{vmatrix}
    \sin\xi\sin\phi\cos\theta&\cos\xi\sin\phi\cos\theta&\cos\phi\cos\theta&-\sin\theta\\
    \sin\xi\sin\phi\sin\theta&\cos\xi\sin\phi\sin\theta&\cos\phi\sin\theta&\cos\theta\\
    \sin\xi\cos\phi&\cos\xi\cos\phi&-\sin\phi&0\\
    \cos\xi&-\sin\xi&0&0\\
  \end{vmatrix}
\end{align*}
我们考虑行列式的拉普拉斯展开,得到
\begin{align*}
&  -\cos\xi \begin{vmatrix}
    \cos\xi\sin\phi\cos\theta&\cos\phi\cos\theta&-\sin\theta\\
\cos\xi\sin\phi\sin\theta&\cos\phi\sin\theta&\cos\theta\\
\cos\xi\cos\phi&-\sin\phi&0
  \end{vmatrix}-\sin\xi \begin{vmatrix}
    \sin\xi\sin\phi\cos\theta&\cos\phi\cos\theta&-\sin\theta\\
\sin\xi\sin\phi\sin\theta&\cos\phi\sin\theta&\cos\theta\\
\sin\xi\cos\phi&-\sin\phi&0\\
  \end{vmatrix}\\&=-\cos^2\xi \begin{vmatrix}
    \sin\phi\cos\theta&\cos\phi\cos\theta&-\sin\theta\\
\sin\phi\sin\theta&\cos\phi\sin\theta&\cos\theta\\
\cos\phi&-\sin\phi&0
  \end{vmatrix}-\sin^{2}\xi \begin{vmatrix}
    \sin\phi\cos\theta&\cos\phi\cos\theta&-\sin\theta\\
\sin\phi\sin\theta&\cos\phi\sin\theta&\cos\theta\\
\cos\phi&-\sin\phi&0
  \end{vmatrix}
\end{align*}
我们只用来计算三阶行列式
\begin{align*}
T=  \begin{vmatrix}
    \sin\phi\cos\theta&\cos\phi\cos\theta&-\sin\theta\\
\sin\phi\sin\theta&\cos\phi\sin\theta&\cos\theta\\
\cos\phi&-\sin\phi&0
  \end{vmatrix}
\end{align*}
即可.在正式计算之前,请允许我节外生枝.我们来看在我们计算三维单位球体积
时出现的三阶 Jacobi 行列式
\begin{align*}
  \begin{vmatrix}
    \frac{\pa x}{\pa\rho}&\frac{\pa x}{\pa\phi}&\frac{\pa
      x}{\pa\theta}\\
    \frac{\pa y}{\pa\rho}&\frac{\pa y}{\pa\phi}&\frac{\pa
      y}{\pa\theta}\\
    \frac{\pa z}{\pa\rho}&\frac{\pa z}{\pa\phi}&\frac{\pa
      z}{\pa\theta}\\
  \end{vmatrix}=
  \begin{vmatrix}
    \sin\phi\cos\theta&\rho\cos\phi\cos\theta&-\rho\sin\phi\sin\theta\\
    \sin\phi\sin\theta&\rho\cos\phi\sin\theta&\rho\sin\phi\cos\theta\\
\cos\phi&-\rho\sin\phi&0\\
  \end{vmatrix}=T\rho^2\sin\phi .
\end{align*}
可见其关系.易得 $T=1$.因此,我们最终算得4阶 Jacobi 行列式为
$$
-\rho^{3}\sin^{2}\xi\sin\phi.
$$
因此,我们可以把上述积分化为
\begin{align*}
\int_0^{2\pi}\int_0^1\int_0^{\pi}\int_0^{\pi}\rho^3\sin^2\xi\sin\phi
 d\phi d\xi d\rho d\theta&=\int_0^{2\pi}\int_0^1\int_0^{\pi}
 2\rho^3\sin^2\xi d\xi d\rho
 d\theta\\&=\int_0^{2\pi}\int_0^1 \pi \rho^3d\rho
 d\theta\\&=\int_0^{2\pi}\frac{1}{4}\pi d\theta\\&=\frac{1}{2}\pi^2.
\end{align*}
\end{proof}
% ----------------------------------------------------------------------------------------
% BIBLIOGRAPHY
% ----------------------------------------------------------------------------------------
  
\bibliographystyle{unsrt}
  
\bibliography{sample}
  
% ----------------------------------------------------------------------------------------
\end{CJK}
\end{document}